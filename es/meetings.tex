\chapter{Reuniones, Reuniones, Reuniones}\label{s:meetings}

La mayoria de la gente es muy mala al organizar reuniones:
no llevan una agenda,
no se toman unos minutos,
hablan vagamente o se desvían en irrelevancias,
dicen algo trivial o repiten lo que otros han dicho
sólo para decir algo,
y mantienen conversaciones secundarias 
(lo cual garantiza que la reunión será una pérdida de tiempo).
Saber cómo organizar una reunión de manera eficiente
es una habilidad central para cualquiera que desee terminar bien las cosas;
saber cómo participar en la reunión de otra persona es igual de importante
(Aunque recibe mucha menos atención --- como dijo una colega una vez,
todos ofrecen entrenamiento de líderes,
pero nadie ofrece entrenamiento de seguidores).

Las reglas más importantes para hacer que las reuniones sean eficientes no son secretas,
pero rara vez se siguen:

\begin{description}

\item[Decide si realmente se necesita una reunión.]
  Si el único propósito es compartir información,
  envía un breve correo electrónico en su lugar.
  Recuerda,
  puedes leer más rápido que cualquiera pueda hablar:
  si alguien tiene datos para que el resto del equipo los tome,
  la forma más educada de comunicarlos es escribirlos.

\item[Escribe una agenda.]
  Si a nadie le importa lo suficiente la reunión como para escribir una lista de puntos
  de lo que se discutirá,
  la reunión probablemente no se necesita.

\item[Incluye horarios en la agenda.]
  Las agendas también pueden ayudarte para evitar que los primeros puntos le roben tiempo a los últimos
  si incluyes el tiempo que le dedicarás a cada punto en la agenda.
  Tus primeras estimaciones con cualquier grupo nuevo serán tremendamente optimistas,
  así que revísalos nuevamente para las siguientes reuniones.
  Sin embargo,
  no deberías planear una segunda o tercera reunión
  porque la primera duró tiempo extra:
  en cambio,
  trata de averiguar por qué ocupaste tiempo extra y arregla el problema que lo originó. 

\item[Prioriza.]
  Cada reunión es un microproyecto,
  por lo tanto el trabajo debería priorizarse de la misma manera que se hace para otros proyectos:
  aquello que tendrá alto impacto pero lleva poco tiempo debería realizarse primero,
  y aquello que tomará mucho tiempo pero tiene bajo impacto debería saltearse.

\item[Haz a una persona responsable de mantener las cosas en movimiento.]
  Una persona debería tener la tarea de mantener los puntos a tiempo,
  llamando la atención a la gente que esté chequeando correo electrónico o  teniendo  conversaciones paralelas,
  pidiendo a aquellos que estan hablando mucho que lleguen al punto,
  e invitando a gente que no interviene que exprese su opinión.
  Esta persona \emph{no} debería hacer toda la charla;
  en realidad, 
  en una reunión bien armada aquel que esté a cargo hablará menos 
  que los otros participantes.

\item[Pide amabilidad.]
  Nadie llegue a ser grosero,
  nadie empiece a divagar,
  y si alguien se va de tema,
  es tanto el derecho como la responsabilidad del moderador decir,
  ``Discutamos eso en otro lado.''

\item[Sin interrupciones.]
  Los participantes deben levantar la mano o poner una nota adhesiva
  si quieren hablar después.
  Si la persona que está hablando no los nota,
  quien modera la reunión debería hacerlo.

\item[Sin tecnología]
  A menos que sea necesario por razones de accesibilidad.
  Insistir que todos guarden sus teléfonos, tabletas, y computadoras de modo amable.
  (p.ej.\ por favor cierren sus aparatos electrónicos).

\item[Registro de minutas.]
  Alguna otra persona que no sea quien modere debería tomar notas de forma puntual sobre 
  los fragmentos más importantes de información que fueron compartidos,
  todas las decisiones tomadas,
  y todas las tareas que se asignaron a alguien.

\item[Toma notas.]
  Mientras otras personas estan hablando,
  los participantes deberían tomar notas de preguntas que quieran hacer o de observaciones que quieran realizar.
  (Te sorprenderás qué inteligente pareces cuando llega tu turno para hablar.)

\item[Termina temprano.]
  Si tu reunión está programada de 10:00-11:00,
  debes buscar de terminar a las 10:50 para dar tiempo a la gente de pasar por el baño 
  en su camino a donde vayan luego.

\end{description}

Tan pronto termina la reunión,
envía a todos un correo electrónico con la minuta o publícala en la web:

\begin{description}

\item[La gente que no estuvo en la reunión puede mantenerse al tanto de lo que ocurrió.]
  Una página web o un mensaje de correo electrónico es una forma mucho más eficiente de ponerse al día
  que preguntarle a un compañero de equipo qué te perdiste.

\item[Cualquiera puede comprobar lo que realmente se dijo o prometió.]
  Más de una vez,
  he revisado la minuta de una reunión en la que estuve
  y pensé, ``Yo dije eso?''
  or, ``Espera un minuto, yo no prometí tenerlo listo para entonces!''
  Accidentalmente o no,
  muchas veces la gente recordará las cosas de manera diferente;
  escribiéndolo da la oportunidad a los miembros del equipo de corregir errores,
  lo que puede ahorrar muchos malos entendidos más tarde.

\item[Las personas pueden ser responsables en reuniones posteriores.]
  No tiene sentido hacer listas de preguntas y puntos de acción
  si después no los sigues.
  Si está utilizando algún tipo de sistema de seguimiento de temas,
  crea un tema por cada pregunta o tarea justo después de la reunión
  y actualiza los que se cumplieron,
  luego comienza cada reunión pasando por una lista de esos temas.

\end{description}

\cite{Brow2007,Broo2016,Roge2018} ten muchos consejos para organizar reuniones.
Según mi experiencia,
una hora de entrenamiento en cómo ser moderador
es una de las mejores inversiones que harás.

\begin{aside}{Notas Adhesivas y Bingo para Interrupción}
  Algunas personas están tan acostumbradas al sonido de su propia voz
  que insistirán en hablar la mitad del tiempo
  sin importar cuántas personas haya en la habitación.
  Para evitar esto,
  entrega a todos tres notas adhesivas al comienzo de la reunión.
  Cada vez que hablen,
  tienen que sacar una nota adhesiva.
  Cuando se queden sin notas,
  no se les permitirá hablar hasta que todos hayan usado al menos una,
  en ese momento todos recuperan todas sus notas adhesivas.
  Esto asegura que nadie hable más de tres veces que
  la persona más callada de la reunión,
  y cambia completamente la dinámica de la mayoría de los grupos:
  personas que dejan de intentar ser escuchadas porque siempre son tapadas
  de repente tienen espacio para contribuir,
  y aquellas que hablaban con demasiada frecuencia se dan cuenta lo injustos que han sido\footnote{
    Yo ciertamente lo hice cuando me hicieron esto{\ldots}
  }.

  Otra técnica is un bingo de interrupción.
  Dibuja una tabla y etiqueta las filas y columnas con los nombres de los participantes.
  Agrega en la celda apropiada una marca para contar 
  cada vez que alguien interrumpa a otro,
  y toma un momento para compartir los resultados a la mitad de la reunión.
  En la mayoría de los casos,
  verás que una o dos personas son las que interrumpen siempre,
  a menudo sin ser conscientes de ello.
  Eso solo muchas veces es suficiente  para detenerlas.
  (Nota que esta técnica está destinada a manejar las interrupciones,
  no el tiempo de conversación:
  puede ser apropiado para gente con mucho conocimiento de un tema
  del que habla seguido en una reunión,
  pero nunca es apropiado cortar repetidamente a las personas.)
\end{aside}

\seclbl{Las reglas de Martha}{s:meetings-marthas-rules}

Las organizaciones de todo el mundo realizan sus reuniones de acuerdo a 
\hreffoot{https://en.wikipedia.org/wiki/Robert\%27s\_Rules\_of\_Order}{Reglas de Orden de Roberto},
pero son mucho más formales que lo requerido para proyectos pequeños.
Una ligera alternativa conocida como ``Las reglas de Marta''
puede que sea mucho mejor para la toma de decisiones por concenso~\cite{Mina1986}:

\begin{enumerate}

\item
  Antes de cada reunión,
  cualquiera que lo desee puede patrocinar una propuesta compartiéndola con el grupo.
  Las propuestas deben ser archivadas al menos 24 horas antes de una reunión para ser consideradas en esa reunión,
  y debe incluir:
  \begin{itemize}
  \item un resumen de una línea;
  \item el texto completo de la propuesta;
  \item cualquier información de antecedentes requerida;
  \item pros y contras; y
  \item posibles alternativas
  \end{itemize}
  Las propuestas deberian ser a lo sumo de 2 páginas.

\item
  Se establece un quórum en una reunión si la mitad o más de los miembros votantes están presentes.

\item
  Una vez que una persona patrocina una propuesta,
  es responsable de ella.
  El grupo no puede discutir o votar sobre el tema a menos que quien patrocina o su delegado esté presente.
  La persona patrocinadora también es responsable de presentar el tema al grupo.

\item
  Después que la persona patrocinadora presente la propuesta,
  se emite un voto preliminar para la propuesta antes de cualquier discusión:
  \begin{itemize}
  \item ¿A quién le gusta la propuesta?
  \item ¿A quién le parece razonable con la propuesta?
  \item ¿Quién se siente incómodo con la propuesta?
  \end{itemize}
  Los votos preliminares se pueden hacer con el pulgar hacia arriba, el pulgar hacia los lados o el pulgar hacia abajo (en persona)
  o escribiendo +1, 0 o -1 en el chat en línea (en reuniones virtuales).

\item
  Si a todos o a la mayoria del grupo le gusta o resulta razonable la propuesta,
  se pasa inmediatamente a una votación formal sin más discusión.

\item
  Si la mayoría del grupo está desconforme con la propuesta,
  se pospone para que la persona patrocinadora pueda volver a trabajar sobre ella.

\item
  Si algunos miembros se sienten desconformes, pueden expresar brevemente sus objeciones.
  Luego se establece un temporizador para una breve discusión moderada por una persona facilitadora.
  Después de diez minutos o cuando nadie más tenga algo que agregar (lo que ocurra primero),
  quien facilita llama para una votación si-o-no sobre la pregunta:
  ``¿Deberíamos implementar esta decisión aun con las objeciones establecidas?''
  Si la mayoría vota ``si'' la propuesta se implementa.
  De lo contrario, la propuesta se devuelve a la persona patrocinadora para trabajarla más.

\end{enumerate}

\seclbl{Reuniones en linea}{s:meetings-online}

\hreffoot{https://chelseatroy.com/2018/03/29/why-do-remote-meetings-suck-so-much/}{Discusión de Chelsea Troy}
de por qué las reuniones en línea son a menudo frustrantes e improductivas 
resulta un punto importante:
en la mayoría de las reuniones en línea,
la primera persona en hablar durante una pausa toma la palabra.
¿El resultado?
``Si tienes algo que quieres decir,
tienes que dejar de escuchar a la persona que está hablando actualmente
y en lugar de eso, enfócate en cuándo van a detenerse o terminar, 
para que puedas saltar sobre ese nanosegundo de silencio y ser el primero en pronunciar algo.
El formato{\ldots}alienta a los participantes que deseen contribuir a decir más y escuchar menos.''

La solución es chatear (charla en texto) a la par de la videoconferencia
donde las personas pueden indicar que quieren hablar,
Quien modere entonces selecciona personas de la lista de espera.
Si la reunión es grande o argumentativa,
mantener a todos silenciados
y solo permitir a quien modere liberar el micrófono a las personas.

\seclbl{La autopsia}{s:meetings-post-mortem}

Cada proyecto debe terminar con una autopsia
en la que los participantes reflexionan sobre lo que acaban de lograr
y qué podrían mejorar la próxima vez.
Su objetivo es \emph{no} señalar con el dedo de la vergüenza a las personas,
aunque si eso tiene que suceder,
la autopsia es el mejor lugar para ello.

Una autopsia se realiza como cualquier otra reunión
con algunas pautas adicionales~\cite{Derb2006}:

\begin{description}

\item[Conseguir una persona que modere y que no fuera parte del proyecto]
  y no tiene interés en eso.

\item[Reservar una hora y solo una hora.]
  Según mi experiencia,
  nada útil se dice en los primeros diez minutos de la primera autopsia de alguien,
  dado que las personas son naturalmente un poco tímidas para alabar o condenar su propio trabajo.
  Igualmente,
  no se dice nada útil después de la primera hora:
  si aún sigues hablando,
  probablemente sea porque una o dos personas
  tienen cosas que quieren quitarse del pecho
  en lugar de sugerencias para mejoras.

\item[Requerir asistencia.]
  Todos los que formaron parte del proyecto deben estar en la sala para la autopsia.
  Esto es más importante de lo que piensas:
  las personas que tienen más que aprender de la autopsia
  en general son menos propensas a presentarse si la reunión es opcional.

\item[Confeccionar dos listas.]
  Cuando estoy moderando,
  pongo los encabezados `` Hazlo otra vez '' y `` Hazlo de manera diferente '' en la pizarra,
  luego pido a cada persona que me dé una respuesta por cada lista en orden 
  sin repetir nada que ya se haya dicho.

\item[Comentar sobre acciones en lugar de individuos.]
  Para cuando el proyecto esté terminado,
  es posible que algunas personas ya no sean amigas.
  No dejes que esto desvíe la reunión:
  si alguien tiene una queja específica sobre otro miembro del equipo,
  pídeles que critiquen un evento o decisión en particular.
  ``Tiene una mala actitud'' \emph{no} ayuda a nadie a mejorar.

\item[Priorizar las recomendaciones.]
  Una vez que los pensamientos de todos estén al descubierto,
  ordénalos según cuáles son los más importantes de mantener,
  y cuáles son los más importantes para cambiar.
  Probablemente solo podrás abordar uno o dos de cada lista en tu próximo proyecto,
  pero si haces eso cada vez,
  tu vida mejorará rápidamente.

\end{description}
