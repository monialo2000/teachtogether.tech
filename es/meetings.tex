\chapter{Reuniones, Reuniones, Reuniones}\label{s:meetings}

La mayoria de la gente es muy mala al hacer reuniones:
no llevan una agenda,
no se toman unos minutos,
hablan vagamente o se desvían en irrelevancias,
dicen algo trivial o repiten lo que ya se ha dicho
sólo para decir algo,
y mantienen conversaciones paralelas 
(lo cual garantiza que la reunión será una pérdida de tiempo).
Saber cómo organizar una reunión de manera eficiente
es una habilidad central para cualquiera que quiera terminar las cosas;
saber cómo participar en la reunión de otra persona es igual de importante
(Aunque recibe mucha menos atención --- como dijo una colega una vez,
todos ofrecen entrenamiento de líderes,
pero nadie ofrece entrenamiento de seguidores).

Las reglas más importantes para hacer que las reuniones sean eficientes no son secretas,
pero rara vez se siguen:

\begin{description}

\item[Decidir si realmente se necesita una reunión.]
  Si el único propósito es compartir información,
  envíe un breve correo electrónico en su lugar.
  Recuerda,
  puedes leer más rápido que cualquiera pueda hablar:
  si alguien tiene datos para que el resto del equipo los tome,
  la forma más educada de comunicarlos es escribirlos.

\item[Escribir una agenda.]
  Si a nadie le importa lo suficiente la reunión como para escribir una lista de puntos
  de lo que se discutirá,
  la reunión probablemente no se necesita.

\item[Incluir horarios en la agenda.]
  Las agendas también pueden ayudarte para evitar que los puntos  anteriores le roben tiempo a los puntos posteriores
  si incluyes el tiempo que le dedicarás a cada punto en la agenda.
  Tus primeras estimaciones con cualquier grupo nuevo serán tremendamente optimistas,
  así que revísalos nuevamente para las siguientes reuniones.
  Sin embargo,
  no deberías planear una segunda o tercera reunión
  porque la primera duró tiempo extra:
  en cambio,
  trata de averiguar por qué ocupaste tiempo extra y arregla el problema que lo originó. 

\item[Priorizar.]
  Cada reunión es un microproyecto,
  por lo tanto el trabajo debería priorizarse de la misma manera que se hace para otros proyectos:
  aquello que tendrá alto impacto pero lleva poco tiempo debería realizarse primero,
  y aquello que tomará mucho tiempo pero tiene bajo impacto debería saltearse.

\item[Hacer a una persona responsable de mantener las cosas en movimiento.]
  Una persona debería tener la tarea de mantener los puntos en tiempo,
  llamando la atención a la gente que esté chequeando correo electrónico o  teniendo  conversaciones fuera de tema,
  pidiendo a aquellos que estan hablando mucho que llegue al punto,
  e invitando a gente que no interviene que exprese su opinión.
  Esta persona \emph{no} debería hacer toda la charla;
  en realidad, 
  en una reunión bien armada, aquel que este a cargo hablará menos 
  que los otros participantes.

\item[Pedir amabilidad.]
  Nadie llegue a ser grosero,
  nadie empiece a divagar,
  y si alguien se va de tema,
  es tanto el derecho como la responsabilidad del moderador decir,
  ``Discutamos eso en otro lado.''

\item[Sin interrupciones.]
  Los participantes deben levantar la mano o pegar una nota
  si quieren hablar después.
  Si el orador no los nota,
  el moderador debería.

\item[Sin tecnología]
  A menos que sea necesario por razones de accesibilidad.
  Insistir que todos guarden sus teléfonos, tabletas, y computadoras de modo amable.
  (p.ej.\ por favor cierren sus aparatos electrónicos).

\item[Registro de minutas.]
  Alguna otra persona, que no sea quien modere, debería tomar notas de forma puntual sobre 
  los fragmentos más importantes de información que fueron compartidos,
  todas las decisiones tomadas,
  y todas las tareas que se asignaron a alguien.

\item[Tomar notas.]
  Mientras otras personas estan hablando,
  los participantes deberían tomar notas de preguntas que quieran hacer o de observaciones que quieran realizar.
  (Te sorprenderás qué inteligente pareces cuando llega tu turno para hablar.)

\item[Termina temprano.]
  Si tu reunión está programada de 10:00-11:00,
  debes buscar de terminar a las 10:50 para dar tiempo a la gente de pasar por el baño 
  en su camino a donde vayan luego.

\end{description}

Tan pronto termina la reunión,
envía a todos un correo electrónico con la minuta o publícala en la web:

\begin{description}

\item[La gente que no estuvo en la reunión puede mantenerse al tanto de lo que ocurrió.]
  Una página web o un mensaje de correo electrónico es una forma mucho más eficiente de ponerse al día
  que preguntarle a un compañero de equipo qué te perdiste.

\item[Cualquiera puede comprobar lo que realmente se dijo o prometió.]
  Más de una vez,
  he revisado la minuta de una reunión en la que estuve
  y pensé, ``Yo dije esto?''
  or, ``Un minuto, yo no prometí tenerlo listo para entonces!''
  Accidentalmente o no,
  muchas veces la gente recordará las cosas de manera diferente;
  escribiéndolo da la oportunidad a los miembros del equipo de corregir errores,
  lo que puede ahorrar muchos malos entendidos más adelante.

\item[People can be held accountable at subsequent meetings.]
  No tiene sentido hacer listas de preguntas y puntos de acción
  si después no los sigues.
  Si está utilizando algún tipo de sistema de seguimiento de temas,
  crea un tema por cada pregunta o tarea sk justo después de la reunión
  y actualiza los que se cumplieron,
  luego comienza cada reunión revisando una lista de esos temas.

\end{description}

\cite{Brow2007,Broo2016,Roge2018} ten muchos consejos para organizar reuniones.
Según mi experiencia,
una hora de entrenamiento en cómo ser moderador
es una de las mejores inversiones que harás.

\begin{aside}{Notas Adhesivas y Bingo para Interrupción}
  Algunas personas están tan acostumbradas al sonido de su propia voz
  que insistirán en hablar la mitad del tiempo
  sin importar cuántas personas haya en la habitación.
  Para evitar esto,
  entrega a todos tres notas adhesivas al comienzo de la reunión.
  Cada vez que hablen,
  tienen que sacar una nota adhesiva.
  Cuando se queden sin notas,
  no se les permitirá hablar hasta que todos hayan usado al menos una,
  en ese momento todos recuperan todas sus notas adhesivas.
  Esto asegura que nadie hable más de tres veces más que
  la persona más callada de la reunión,
  y cambia completamente la dinámica de la mayoría de los grupos:
  personas que dejan de intentar ser escuchadas porque siempre son tapadas
  de repente tienen espacio para contribuir,
  y aquellas que hablaban demasiado rápidamente se dan cuenta lo injustos que han sido\footnote{
    Ciertamente lo hice cuando me hicieron esto{\ldots}
  }.

  Otra técnica is un bingo de interrupción.
  Dibuja una tabla y etiqueta las filas y columnas con los nombres de los participantes.
  Agrega en la celda apropiada una marca para contar 
  cada vez que alguien interrumpe a otro,
  y toma un momento para compartir los resultadosa la mitad de la reunión.
  En la mayoría de los casos,
  verás que una o dos personas son las que interrumpen siempre,
  a menudo sin ser conscientes de ello.
  Eso solo muchas veces es suficiente  para detenerlas.
  (Note que esta técnica está destinada a manejar las interrupciones,
  no el tiempo de conversación:
  puede ser apropiado para gente con mucho conocimiento de un tema
  que habla seguido de él en una reunión,
  pero nunca es apropiado cortar repetidamente a las personas.)
\end{aside}

\seclbl{Las reglas de Martha}{s:meetings-marthas-rules}

Las organizaciones de todo el mundo realizan sus reuniones de acuerdo a 
\hreffoot{https://en.wikipedia.org/wiki/Robert\%27s\_Rules\_of\_Order}{Reglas de Orden de Roberto},
pero son mucho más formales que lo requerido para proyectos pequeños.
Una ligera alternativa conocida como ``Las reglas de Marta''
puede que sean mucho mejor para la toma de decisiones por concenso~\cite{Mina1986}:

\begin{enumerate}

\item
  Before each meeting,
  anyone who wishes may sponsor a proposal by sharing it with the group.
  Proposals must be filed at least 24 hours before a meeting in order to be considered at that meeting,
  and must include:
  \begin{itemize}
  \item a one-line summary;
  \item the full text of the proposal;
  \item any required background information;
  \item pros and cons; and
  \item possible alternatives
  \end{itemize}
  Proposals should be at most two pages long.

\item
  A quorum is established in a meeting if half or more of voting members are present.

\item
  Once a person has sponsored a proposal,
  they are responsible for it.
  The group may not discuss or vote on the issue unless the sponsor or their delegate is present.
  The sponsor is also responsible for presenting the item to the group.

\item
  After the sponsor presents the proposal,
  a preliminary vote is cast for the proposal prior to any discussion:
  \begin{itemize}
  \item Who likes the proposal?
  \item Who can live with the proposal?
  \item Who is uncomfortable with the proposal?
  \end{itemize}
  Preliminary votes can be done thumb up, thumb sideways, or thumb down (in person)
  or by typing +1, 0, or -1 into online chat (in virtual meetings).

\item
  If all or most of the group likes or can live with the proposal,
  it is immediately moved to a formal vote with no further discussion.

\item
  If most of the group is uncomfortable with the proposal,
  it is postponed for further rework by the sponsor.

\item
  If some members are uncomfortable they can briefly state their objections.
  A timer is then set for a brief discussion moderated by the facilitator.
  After ten minutes or when no one has anything further to add (whichever comes first),
  the facilitator calls for a yes-or-no vote on the question:
  ``Should we implement this decision over the stated objections?''
  If a majority votes ``yes'' the proposal is implemented.
  Otherwise, the proposal is returned to the sponsor for further work.

\end{enumerate}

\seclbl{Reuniones en linea}{s:meetings-online}

\hreffoot{https://chelseatroy.com/2018/03/29/why-do-remote-meetings-suck-so-much/}{Chelsea Troy's discussion}
of why online meetings are often frustrating and unproductive
makes an important point:
in most online meetings,
the first person to speak during a pause gets the floor.
The result?
``If you have something you want to say,
you have to stop listening to the person currently speaking
and instead focus on when they're gonna pause or finish
so you can leap into that nanosecond of silence and be the first to utter something.
The format{\ldots}encourages participants who want to contribute to say more and listen less.''

The solution is to run a text chat beside the video conference
where people can signal that they want to speak,
The moderator then selects people from the waiting list.
If the meeting is large or argumentative,
have everyone mute themselves
and only allow the moderator to unmute people.

\seclbl{The Post Mortem}{s:meetings-post-mortem}

Every project should end with a post mortem
in which participants reflect on what they just accomplished
and what they could do better next time.
Its aim is \emph{not} to point the finger of shame at individuals,
although if that has to happen,
the post mortem is the best place for it.

A post mortem is run like any other meeting
with a few additional guidelines~\cite{Derb2006}:

\begin{description}

\item[Get a moderator who wasn't part of the project]
  and doesn't have a stake in it.

\item[Set aside an hour, and only an hour.]
  In my experience,
  nothing useful is said in the first ten minutes of anyone's first post mortem,
  since people are naturally a bit shy about praising or damning their own work.
  Equally,
  nothing useful is said after the first hour:
  if you're still talking,
  it's probably because one or two people
  have things they want to get off their chests
  rather than suggestions for improvements.

\item[Require attendance.]
  Everyone who was part of the project ought to be in the room for the post mortem.
  This is more important than you might think:
  the people who have the most to learn from the post mortem
  are often least likely to show up if the meeting is optional.

\item[Make two lists.]
  When I'm moderating,
  I put the headings ``Do Again'' and ``Do Differently'' on the board,
  then ask each person to give me one item for each list in turn
  without repeating anything that has already been said.

\item[Comment on actions rather than individuals.]
  By the time the project is done,
  some people may no longer be friends.
  Don't let this sidetrack the meeting:
  if someone has a specific complaint about another member of the team,
  require them to criticize a particular event or decision.
  ``They had a bad attitude'' does \emph{not} help anyone improve.

\item[Prioritize the recommendations.]
  Once everyone's thoughts are out in the open,
  sort them according to which are most important to keep doing
  and which are most important to change.
  You will probably only be able to tackle one or two from each list in your next project,
  but if you do that every time,
  your life will quickly get better.

\end{description}
